\documentclass{article}

\usepackage{minted}

\newminted{python}{frame=lines}

\title{ETB Wrappers}
\author{Gr\'egoire Hamon, Ian A. Mason}

\begin{document}

\maketitle

This document describes the integration of a tool with the {\it
  Evidential Tool Bus} (ETB). This integration consists in writing a
wrapper for the tool, the wrapper exposes the tool to the ETB in the
form of predicates.

\section{Introduction}

A tool wrapper for the ETB is a Python class. Some methods of the
class are declared as {\it predicates} and are seen as such by the
ETB. Predicates can be implemented directly in Python or make calls to
external tools or libraries. Being regular Python classes, wrappers
can contain any code beside predicates.

The ETB provides support to write tools wrapper in the form of a
Python library. This document describes the use of this library. We
consider external command-line tools as examples, but the same
principles apply to libraries.

\section{Tools and Predicates}

A tool is something that the ETB can call to establish or verify a
claim. It can be an external command-line tool, a library, or a Python
function. 

A tool wrapper is a collection of predicates. Predicates from the
Python point of view, are methods of the wrapper class that have a
predicate specification.

\subsection{The {\tt Tool} Class}

A tool wrapper should inherit from the {\tt Tool} class. This class
defines a decorator for predicate specification, as well as generic
utility functions to get the predicates associated with a tool, get
the signature of a predicate, or report an error within the tool to
the ETB. Higher level classes are also available, for example {\tt
  BatchTool} or {\tt InteractiveTool}.

\subsection{Predicate Specification}

A predicate is a method of the wrapper class that is decorated using
{\tt Tool.predicate}. The {\tt Tool.predicate} decorator takes as
argument a string representation of the predicate's signature.

The signature is a comma separated list of arguments to the
predicate. Each argument is of the form {\tt <sign><name>: <kind>}
where {\tt <sign>} is either {\tt +} or {\tt -} and indicates whether
the argument is read or written, {\tt <name>} is a variable name, and
{\tt <kind>} is the kind of this variable and can be either {\tt
  value}, {\tt file} or {\tt handle}. For example, the signature of a
predicate {\tt bmc} that checks if a property {\tt P} is valid for a
model {\tt M} up to depth {n} could be:
\begin{pythoncode}
@Tool.predicate('+property: value, +model: file, +depth: value,
                 -valid: value')
\end{pythoncode}
Note that the specification only indicates the kind of the predicates
arguments, not their type. This is used to make sure that files are
synchronized between ETB nodes, or that handles are consistent. Values
are not interpreted or validated by the ETB, this is left to the tool.

\section{Batch Tools}

Batch tools are the easiest to plug into the ETB. These are tools for
which each predicate can be invoked in isolation.

\subsection{The {\tt BatchTool} Class}

The {\tt BatchTool} class in the module {\tt structs} can be used to
write a wrapper for a batch tool. The class offers infrastructure to
call a tool as an external process or parse the results coming from
that external process. The main method is {\tt run}:
\begin{pythoncode}
  wrapper.run(result, *args)
\end{pythoncode}
calls the command specified by {\tt *args}, parses the output and
returns it in {\tt result}. Intermediate methods are called to call
the tool ({\tt callTool}) parse ({\tt parseResult}) or interpret ({\tt
  bindResult}) the output and can be overriden as needed.

\subsection{Example: A {\tt SAL} Batch Wrapper}

We would like to write a wrapper for the SAL tool suite. The suite is
composed of a set of batch tools with very similar interfaces: most
tools are called with a set of flags, the name of a SAL context, which
is also the name of a sal file, and often the name of a property.

We start a new class, {\tt SalBatch}, which is a {\tt BatchTool}:
\begin{pythoncode}
class SalBatch(BatchTool):
  """Batch wrapper for SAL"""
\end{pythoncode}

We first define a predicate to call the SAL symbolic model-checker,
{\tt sal-smc}. Our predicate has the following signature:
\begin{pythoncode}
  @Tool.predicate("+Model:file, +Prop:value, -Result:value")
\end{pythoncode}
The predicates states that analyzing the property {\tt prop} for {\tt
  model} gives {\tt result}. {\tt model} is the name of a SAL file,
{\tt prop} is the name of the property to be analyzed, and {\tt
  result} is a variable. 

We can now write the code for the predicate:
\begin{pythoncode}
  def sal_smc(self, model, prop, result):
    sal_smc = 'sal-smc'
    (context, _) = os.path.splitext(model)
    return self.run(result, sal_smc, context, prop)
\end{pythoncode}
We first define the command that we want to call, and prepare its
arguments: The command is simply the string {\tt 'sal-smc'}, we
extract the name of the context from the file name by removing its
extension. Finally, we can call the method {\tt run}, and return its
result: it first invoke the command with the arguments {\tt context}
and {\tt prop}. It then parses its output and binds the result to {\tt
  result}. This process invokes three methods of the class {\tt
  BatchTool} that can be redefined as needed: {\tt callTool} call the
commands and returns its standard output as a string, {\tt
  parseResult} takes that string and produces the result in the
desired format, and {\tt bindResult} binds it to the result
variable. In this example, we use the default implementation of these
methods.

Very similarly, we can add a predicate to check for deadlocks:
\begin{pythoncode}
@Tool.predicate("+model:file, +module:value, -result:value")
def sal_deadlock_check(self, model, module, result) :
  ctx = self.context_from_filename(model)
  return self.run(result, 'sal-deadlock-checker', ctx, module)
\end{pythoncode}
This is exactly as before, we factorized the extraction of the context
name in a method {\tt context\_from\_filename}. Following this model,
we can wrap all the SAL batch commands.

\section{Returning Results from Tools}

In addition to the  {\tt Tool} class is the {\tt Result} class hierarchy,
which is designed to encapsulate the values returned from external tools.

The most common way of returning information to the ETB is by way of a 
substitution, or binding, of values to variables. A substitution is
simply a list of bindings, obtained using  {\tt bindResult}, and returned
to the ETB using the {\tt Substitutions} class.

\begin{pythoncode}
    @Tool.predicate('+a: value, +b: value, -res: value')
    def plus(self, a, b, res):
        a = a.get_val()
        b = b.get_val()
        if res.is_var():
            return Substitutions(self, [self.bindResult(res, a+b)])
        else:
            res = res.get_val()
            return Success(self) if res == a + b else Failure(self)
\end{pythoncode}

The first argument to the {\tt Substitutions} constructor is the tool itself,
the second is a list of substitutions. The interpretation being that 
each substitution satisfies the goal represented by the tool wrapper call.

{\tt Success} and {\tt Failure} are simply convenient abbreviations
for 

\begin{pythoncode}
Substitutions(self, [{}])
Substitutions(self, [])
\end{pythoncode}
respectively.


An example of returning a non-trivial list of substitutions is given
by the {\tt in\_range} example, that for each element in the
range  {\tt [low, up]} returns a substitution binding {\tt res}
to that element.

\begin{pythoncode}
    @Tool.predicate("+low: value, +up: value, -res: value")
    def in_range(self, low, up, res):
        """Result in [low, up] range."""
        low = int(low.val)
        up = int(up.val)
        if low > up:
            return Failure(self)
        if res.is_var():
            slist =  [self.bindResult(res, i) for i in range(low, up+1)]
            return Substitutions(self, slist)
        else:
            res = int(res.val)
            if low <= res <= up:
                return Success(self)
            else:
                return Failure(self)
\end{pythoncode}





\section{Interactive Tools}

Interactive tools require more machinery as the ETB needs to keep a
session open with the tool, and needs to keep track of what is
happening in a given session so that a claim can be replayed.

The communication between the ETB and the tool is also more involved,
as the ETB needs to know when a tools is communicating with it.

\subsection{The {\tt InteractiveTool} Class}

The {\tt InteractiveTool} class can be used as a starting point to
wrap interactive tools. It offers methods to:
\begin{itemize}
\item connect to a tool running as an external command {\tt connect}
\item communicate with this tools through {\tt stdin} ({\tt write} and
  {\tt stdout} ({\tt read})
\item takes care of managing sessions {\tt updateSession}
\end{itemize}
The default implementation makes assumptions about the protocol used
to communicate with the tool, but as for the batch case, these
assumption are encoded within methods that can be overridden:
\begin{itemize}
\item every message send to the tool is followed by a response
\item the response is read line by line
\item a response is complete if the method {\tt parse\_stop} applied
  to the last line read returns true (by default the empty string)
\item the method {\tt parse\_error} takes a response and raise an
  exception is the response contains an error message. It returns {\tt
    None} otherwise (default: always return {\tt None})
\item the method {\tt parse\_output} transform the response in a form
  suitable for the ETB (default: identity)
\end{itemize}

\subsection{Example: {\tt sal-sim}}

As a running example, we are going to write a wrapper for the SAL
simulator. The simulator is started through the command {\tt sal-sim},
then lets the user step through simulations of a model through a
read-eval loop. The wrapper is an {\tt InteractiveTool}:
\begin{pythoncode}
class SALsim(InteractiveTool):
  """ETB wrapper around SALsim."""
\end{pythoncode}
The first thing to do is to start the tool, and get a handle to
communicate with it. For {\tt sal-sim}, we define the following
method, which calls the command, then import the model we want to
simulate, starts the simulation and returns a handle to the ETB:
\begin{pythoncode}
  @Tool.predicate("+Model: file, +Module: value, -Session: handle")
  def salsim_start(self, model, module, session) :
    (ctx, ext) = os.path.splitext(model)
    s = self.connect('sal-sim')
    self.write(s, '(import! "%s")\n' % ctx)
    self.write(s, '(start-simulation! "%s")\n' % module)
    return  Substitutions(self, [ self.bindResult(session, s) ])
\end{pythoncode}
The {\tt connect} method of {\tt InteractiveTool} call the command,
registers the session and gives us a handle to the session back. Using
this handle, we can call the method {\tt write} to send commands to
the tool.

The next predicates lets us get the current states of the system, by
calling the simulator command {\tt (display-curr-states)}:
\begin{pythoncode}
    @Tool.predicate("+Session: handle, -States: value")
    def salsim_current_states(self, session, states) :
        out = self.write(session, '(display-curr-states)\n')
        return  Substitutions(self, [ self.bindResult(states, out) ])
\end{pythoncode}
We send the command to {\tt sal-sim} and return the output as-is. Note
that the output of that command is a textual representation of the
states, and would probably need to be parsed, which we are not doing
here.

We now move on to a more complex predicate, which is changing the
session:
\begin{pythoncode}
    @Tool.predicate("+SessionIn: handle, -SessionOut: handle")
    def salsim_step(self, sessionIn, sessionOut) :
        self.write(sessionIn, '(step!)\n')
        slist = [ self.bindResult(sessionOut, self.updateSession(sessionIn)) ]
        return Substitutions(self, slist)
\end{pythoncode}
{\tt salsim\_step} causes the running simulation to take a step. The
predicates takes the session to advance as argument and returns a new
session. The input session becomes unavailable. This is implemented by
calling the command, then using the {\tt updateSession} method.

Finally, we can have a predicate to close the session:
\begin{pythoncode}
    @Tool.predicate("+Session: handle")
    def salsim_close_session(self, session) :
        self.write(session, '(exit)\n')
        return Failure(self)
\end{pythoncode}

\section{Registering Tools and Rules}

\subsection{Registering a Tool}

Once a wrapper class has been written for a tool, it still needs to be
instantiated and registered with the ETB. A wrapper should be in a
Python file that contains a {\tt register} function. This function is
executed by the ETB when the file is loaded. As an example, to
register the {\tt SalBatch} wrapper, we write:
\begin{pythoncode}
def register(etb):
  "Register the tool"
  etb.add_tool(SalBatch(etb))
\end{pythoncode}


\subsection{Generating Queries}

As a prequel to describing the ability to add rules dynamically,
we illustrate the ability to generate and add simple queries
to the ETB system.
In this example we have two mutually {\em recursive} predicates
{\tt ping(M)} and {\tt pong(M)}, given by the two sets of principles:
\begin{verbatim}
ping(0) :- 
ping(M + 1) :- pong(M) 

pong(0) :- 
pong(M + 1) :- ping(M) 
\end{verbatim}
Note that these are not ETB rules because a predicate cannot
be both a wrapper and the {\em head} of a rule. So we must use
the ability for wrappers to generate queries to achieve the
same effect.


To return a set of dynamic subqueries from a goal {\tt G} we provide the {\tt Queries} class.
\begin{pythoncode}
  Queries(tool, slist, qlist)
\end{pythoncode}
where  {\tt slist} should be a list of {\tt N} substitutions
\begin{verbatim}
[s0, s1, ... , sN]
\end{verbatim}
and {\tt qlist} is a list of {\tt M} queries.
\begin{verbatim}
[q0, q1, ... , qM]
\end{verbatim}
This has the effect of adding the rules:
\begin{verbatim}
si(G) :- si(qj)
\end{verbatim}
for each {\tt si}, and {\tt qj} in the two lists.

As an illustration we provide the string version of {\tt ping},
and the term version of {\tt pong}

\begin{pythoncode}
@Tool.predicate('+n: value')
def ping(self, n):
    n = int(n.val)
    if n == 0 :
        return Success(self)
    else :
        return Queries(self, [{}], [ 'pong(%s)' % str(n-1) ] )
\end{pythoncode}

\begin{pythoncode}
@Tool.predicate('+n: value')
def pong(self, n):
    n = int(n.val)
    if n == 0 :
        return Success(self)
    else :
        return Queries(self, [{}], [ etb.terms.mk_apply("ping", [n-1]) ] )
\end{pythoncode}




\subsection{Adding Rules to the ETB}

The ETB includes the ability to generate and add rules dynamically.
We use the, albeit artificial, {\tt verycomposite} example to showcase this
use of wrappers to return new (temporary or pending) rules. 

We say an integer {\tt M} is {\em composite} if it is not prime, written
{\tt comp(M)}.
A number {\tt M} is said to be {\tt N-verycomposite}  if
{\tt comp(M)}, {\tt comp(M + 1)}, ...., {\tt comp(M + N)}.
We write  {\tt verycomposite(M, N)} to express that {\tt M} is  {\tt N-verycomposite}.
Some simple examples are 
{\tt verycomposite(8, 3)}, {\tt verycomposite(24, 5)}, and 
{\tt verycomposite(90, 7)}.

The general principle is of the form:
\begin{verbatim}
verycomposite(M, N) :- comp(M), comp(M + 1), ..., comp(M + N) .
\end{verbatim}
which is not a rule, but a rule scheme for each fixed {\tt M}. We can use the
ability to generate rules dynamically to achieve the same effect.

To return a set of dynamic rules from a goal {\tt G} we provide the {\tt Lemmata} class.
\begin{pythoncode}
  Lemmatta(tool, slist, llist)
\end{pythoncode}
where  {\tt slist} should be a list of {\tt N} substitutions
\begin{verbatim}
[s0, s1, ... , sN]
\end{verbatim}
and {\tt llist} is a list of {\tt N} lemmattas
\begin{verbatim}
[l0, l1, ... , lN].
\end{verbatim}
each lemmattas itself {\tt li} being a list terms. 
\begin{verbatim}
li = [q0i, ... , qMi]
\end{verbatim}
Using this form results in {\tt N} rules being added to the ETB framework.
Each rule being of the form:
\begin{verbatim}
si(G) :- si(q0i), ..., si(qMi)
\end{verbatim}

We give two examples in the wrappers {\tt verycomposite} and {\tt verycompositeT}
to illustrate the API. We begin with the simple wrapper {\tt comp}

\begin{pythoncode}
@Tool.predicate('+n: value')
 def comp(self, n):
     n = abs(int(n.val))
     if isPrime(n):
         return Failure(self)
     else:
         return Success(self)
\end{pythoncode}

that relies on the predicate {\tt isPrime}. 


%     Lemmatta(tool, slist, llist)
%     slist should be a substitution list: [s0, s1, ... , sN]
%     - the substitution can either be a dict or a terms.Subst object
%     llist should be a list of lemmatta: [l0, l1, ... , lN]
%     - the lists should be the same length
%     - each li is itself a list: [qi0, qi1, ... qiM] where each qi0 is either a terms.Term or a string.

The first version uses the string representation of terms, these are parsed behind the scenes 
by the {\tt Lemmatta} class.
\begin{pythoncode}
@Tool.predicate('+n: value, +m: value')
def verycomposite(self, n, m):
    n = abs(int(n.val))
    m = abs(int(m.val))
    termlist = [ "comp(%s)" % i  for i in range(n, n + m)]
    return Lemmatta(self, [{}], [ termlist ])
\end{pythoncode}

Using the string representation is not mandatory. If it were more convenient to use terms,
they could be provoded instead, as they are in {\tt verycompositeT}:

\begin{pythoncode}
@Tool.predicate('+n: value, +m: value')
def verycompositeT(self, n, m):
    n = abs(int(n.val))
    m = abs(int(m.val))
    lemmatta = [ etb.terms.mk_apply("comp", [i])  for i in range(n, n + m)]
    return Lemmatta(self, [{}], [ lemmatta ])
\end{pythoncode}


\end{document}
